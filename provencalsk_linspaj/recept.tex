\documentclass[a4wide, 10pt]{article}
\usepackage[swedish]{babel}
\usepackage[T1]{fontenc}
\usepackage[utf8]{inputenc}
\usepackage{setspace}
\usepackage{cuisine}
\usepackage{url}

\renewcommand*{\recipetitlefont}{\large\bfseries\sffamily}
\renewcommand*{\recipequantityfont}{\sffamily\bfseries}
\renewcommand*{\recipeunitfont}{\sffamily}
\renewcommand*{\recipeingredientfont}{\sffamily}
\renewcommand*{\recipefreeformfont}{\itshape}

\begin{document}

\begin{recipe}{Linspaj från Provencal}{4 personer}{30 minuter}

\ingredient[3]{dl}{vetemjöl}
\ingredient[100]{g}{smör}
\ingredient[0.5]{dl}{färsk timjan}
\ingredient[2-3]{msk}{vatten}

Tärna smöret. Hacka eller klipp timjan. Blanda ihop med mjöl och lite vatten.
Knåda degen och tryck ut i en pajform. Grädda i ugnen 225$^o$ ca 15 minuter tills
den fått fin färg. Ta ur ugnen och låt svalna.

\ingredient[1.5-2]{dl}{svarta eller gröna linser}
\ingredient[10]{}{torkade tomater}

Klipp de torkade tomaterna i strimlor. Koka upp linserna och stjälp i tomaterna
efter ett tag. Koka tills linserna är färdiga (enligt förpackning). Häll av vattent.

\ingredient[0.5-1]{dl}{olivolja}
\ingredient[1]{msk}{balsamvinäger}
\ingredient{}{salt och peppar}

Blanda ihop dressingen och häll sedan över linserna och blanda.

\ingredient[250]{g}{philadelphiaost}
\ingredient{}{solrosfrön}
\ingredient[1-2]{}{avokado}

Smeta ut osten i botten på pajskalet. Fördela linserna över. Strö solrosfrön
ovanpå. Klyfta avokadon och lägg i mönster på pajen. Klart!

\end{recipe}

(Från Risenta)

\end{document}

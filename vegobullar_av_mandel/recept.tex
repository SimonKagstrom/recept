\documentclass[a4wide, 10pt]{article}
\usepackage[swedish]{babel}
\usepackage[T1]{fontenc}
\usepackage[utf8]{inputenc}
\usepackage{setspace}
\usepackage{cuisine}
\usepackage{url}

\renewcommand*{\recipetitlefont}{\large\bfseries\sffamily}
\renewcommand*{\recipequantityfont}{\sffamily\bfseries}
\renewcommand*{\recipeunitfont}{\sffamily}
\renewcommand*{\recipeingredientfont}{\sffamily}
\renewcommand*{\recipefreeformfont}{\itshape}

\begin{document}

\begin{recipe}{Vegobullar av mandel}{4 personer}{1 timme}

\ingredient[100]{g}{sötmandel}
\ingredient[1 \fr12]{dl}{finriven ost}
\ingredient[1 \fr12]{dl}{ströbröd}
\ingredient[2]{msk}{gul lök}
\ingredient[1]{tsk}{salt}
\ingredient[1]{krm}{svartpeppar}
\ingredient[3]{st}{ägg}
\ingredient[2]{msk}{vatten}

Mixa sötmandeln med stavmixer. Blanda allt och forma till små bollar.

\ingredient[1]{l}{grönsaksbuljong}
\ingredient[]{}{smör/olja}

Koka bullarna i grönsaksbuljong i 6-7 minuter. Stek därefter i smör
eller olja.
\end{recipe}
\end{document}

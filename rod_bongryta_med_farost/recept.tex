\documentclass[a4wide, 10pt]{article}
\usepackage[swedish]{babel}
\usepackage[T1]{fontenc}
\usepackage[utf8]{inputenc}
\usepackage{setspace}
\usepackage{cuisine}
\usepackage{url}

\renewcommand*{\recipetitlefont}{\large\bfseries\sffamily}
\renewcommand*{\recipequantityfont}{\sffamily\bfseries}
\renewcommand*{\recipeunitfont}{\sffamily}
\renewcommand*{\recipeingredientfont}{\sffamily}
\renewcommand*{\recipefreeformfont}{\itshape}

\begin{document}

\begin{recipe}{Röd böngryta med fårost}{4 personer}{30 minuter}

\ingredient[2]{}{rödlökar}
\ingredient[1]{}{stor morot}
\ingredient[2]{}{röda paprikor}
\ingredient[1]{bit}{röd chilipeppar}
\ingredient[2]{msk}{olivolja}

Skala och hacka löken. Skär moroten i slantar. Strimla paprika och
chilipeppar (utan fröväggar).

Hetta upp olivoljan och fräs lök, chili, paprika och morot lätt.

\ingredient[7]{dl}{kokta bönor}
\ingredient[1]{brk}{tomater (400g)}
\ingredient[200]{g}{fårost}
\ingredient{}{salt}
\ingredient{}{basilika/persilja}

Häll på bönor och tomater, späd eventuellt med lite vatten. Salta och
peppra. Låt puttra under lock i ca 10 minuter.

Rör ner lite finhackad basilika/persilja. Smula över fårost och strö
över några hela basilikablad.

\end{recipe}

Lättrostat bröd med olivolja och bulgur eller quinoa är gott till. Två
förpackningar färdigkokta bönor ger lagom mängd, svarta bönor och
kidneybönor tycker jag passar bra i den här rätten.

(Från ``Det vegetariska köket'' av Lotta Brinck och Nisse Peterson)

\end{document}

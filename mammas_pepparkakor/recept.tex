\documentclass[a4wide, 10pt]{article}
\usepackage[swedish]{babel}
\usepackage[T1]{fontenc}
\usepackage[utf8]{inputenc}
\usepackage{setspace}
\usepackage{cuisine}
\usepackage{url}

\renewcommand*{\recipetitlefont}{\large\bfseries\sffamily}
\renewcommand*{\recipequantityfont}{\sffamily\bfseries}
\renewcommand*{\recipeunitfont}{\sffamily}
\renewcommand*{\recipeingredientfont}{\sffamily}
\renewcommand*{\recipefreeformfont}{\itshape}

\begin{document}

\begin{recipe}{Ingalisas pepparkakor}{Tillräckligt för hela julen}{1 timme + 1 förmiddag}
\freeform Mammas recept på pepparkakor, från en kollega på Umeå lasarett 1978.
1 kkp (kaffekopp) är 1.5dl.

\ingredient[1]{kkp}{vatten}
\ingredient[2]{kkp}{sirap (ljus)}
\ingredient[5]{kkp}{socker}
\ingredient[1]{msk}{nejlikor (malda)}
\ingredient[2]{msk}{ingefära}
\ingredient[2]{msk}{kanel}
\ingredient[\fr12]{kg}{smör}
\ingredient[1]{kkp}{grädde}

Blanda vatten, sirap, socker, nejlikor, ingefära och kanel i en stor
kastrull. Koka upp så att det bubblar i någon minut. Tillsätt
smör, låt stå tills det kallnar. Tillsätt grädde och rör
om.

\ingredient[2]{msk}{bikarbonat}
\ingredient[~1.7]{liter}{vetemjöl}

Blanda samman vetemjöl och bikarbonat och tillsätt mjölet och
bikarbonaten i resten av smeten. Låt smeten stå i kylen till nästa
dag.
\end{recipe}

Använd mjöl till utbakningen. Kavla ut degen och använd
pepparkaksformar för att göra pepparkakorna. Lägg på bakplåtspapper
och ställ in plåten med pepparkakorna i ugnen några minuter,
200$^o$C. Pepparkakorna ska sluta bubbla upp och få lite färg. Prova
en kaka först!
\end{document}

\documentclass[a4wide, 10pt]{article}
\usepackage[swedish]{babel}
\usepackage[T1]{fontenc}
\usepackage[utf8]{inputenc}
\usepackage{setspace}
\usepackage{cuisine}
\usepackage{url}

\renewcommand*{\recipetitlefont}{\large\bfseries\sffamily}
\renewcommand*{\recipequantityfont}{\sffamily\bfseries}
\renewcommand*{\recipeunitfont}{\sffamily}
\renewcommand*{\recipeingredientfont}{\sffamily}
\renewcommand*{\recipefreeformfont}{\itshape}

\begin{document}

\begin{recipe}{Morotsgratäng med pesto}{5-6 portioner}{1 timme}

\ingredient[7 \fr12]{dl}{mjölk}
\ingredient[1 \fr23]{dl}{creme fraiche}
\ingredient[1]{tsk}{krossad svartpeppar}

Blanda mjölk och creme fraiche i en skål, peppra.

\ingredient[50]{g}{smör}
\ingredient[1 \fr14]{dl}{vetemjöl}

Smält smöret i en stor kastrull och tillsätt mjölet. Rör på svag värme
tills blandningen bubblar litegrand.

Tillsätt mjölkblandningen lite i taget, låt koka upp, och tillsätt mer
under ständig omrörning. Såsen skall vara slät mellan varje omgång.

\ingredient[~150]{g}{cheddar eller chevre}
\ingredient[4]{st}{ägg}

Ta såsen från värmen och tillsätt smulad/riven ost. Låt svalna något.

Vispa äggen och tillsätt lite i taget till såsen under omrörning. Ta
undan \fr13 av såsen till topplagret.

\ingredient[2]{msk}{pesto}
\ingredient[~500]{g}{morötter}
\ingredient[2]{st}{palsternackor}

Riv morötter och palsternackor. Blanda i såsen tillsammans med pesto.

\ingredient[~10]{st}{lasagneplattor}
\ingredient[50]{g}{riven ost}

Lägg lasagneplattor i en ugnsform. Fördela morots- och såsblandningen
på plattorna i tre lager. Fördela den sparade såsen på det sista
lagret lasagneplattor. Låt vila i ca 15 minuter.

Sätt ugnen på 150\0C, gratinera i ungefär 40 minuter tills såsen
stelnat och ytan fått fin färg.
\end{recipe}

(Från Stora vegetarisk kokboken, modifierat)

\end{document}

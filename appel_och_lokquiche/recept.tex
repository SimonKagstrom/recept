\documentclass[a4wide, 10pt]{article}
\usepackage[swedish]{babel}
\usepackage[T1]{fontenc}
\usepackage[utf8]{inputenc}
\usepackage{setspace}
\usepackage{cuisine}
\usepackage{url}

\renewcommand*{\recipetitlefont}{\large\bfseries\sffamily}
\renewcommand*{\recipequantityfont}{\sffamily\bfseries}
\renewcommand*{\recipeunitfont}{\sffamily}
\renewcommand*{\recipeingredientfont}{\sffamily}
\renewcommand*{\recipefreeformfont}{\itshape}

\begin{document}

\begin{recipe}{Äppel och lökquiche}{4 personer}{1 timme}

\ingredient[190]{g}{gruyereost}
\ingredient[3 \fr34]{dl}{mjöl}
\ingredient[]{}{salt}
\ingredient[\fr12-1]{msk}{senap}
\ingredient[2-3]{msk}{vatten}
\ingredient[75]{g}{smör/margarin}

Blanda 75g av osten med mjöl, lite salt, senapspulver, vatten och
margarin i en bunke. Knåda ihop till en slät deg och ställ i kylen.

\ingredient[1]{st}{gul lök}
\ingredient[400]{g}{äpplen}
\ingredient[25]{g}{smör}

Skala och hacka löken. Riv äpplena. Värm smör i en kastrull och stek
lök tills den är genomskinlig. Rör sedan ner äpplena och fräs i 2-3
minuter. Ta från värmen.

\ingredient[1-2]{msk}{senap}
\ingredient[2-3]{st}{ägg}
\ingredient[1 \fr12]{dl}{grädde}
\ingredient[1]{tsk}{blandade torkade örter}
\ingredient[\fr12]{knippe}{persilja}
\ingredient[]{}{svartpeppar}

Rör ihop ägg, grädde, örter, senap och persilja i en bunke. Vispa
tills det skummar. Riv ner resten av osten i blandningen. Peppra.
\end{recipe}

Tryck ut pajskalet i en form. Blanda ner äppelfräs och äggstanning i
skalet. Skär några skivor ost med osthyvel och lägg ovanpå
pajen. Grädda i ugnen i ~20 minuter på 190 grader.
\end{document}
